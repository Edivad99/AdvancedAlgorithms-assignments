\section{Program's features and measurements}

\subsection{Program's features}

\subsubsection{Introduction and setting up}
The program was written entirely in C\# mainly for convenience, in fact from the standard library, it's possible to 
use several useful data structures such as \textbf{PriorityQueue} used in Prim's algorithm.
Before running the program, follow these steps:
\begin{enumerate}
    \item Install \textbf{.NET 6.0};
    \item Be sure to change the \verb|PATH_FOLDER| variable inside \textit{Program.cs} with the own relative path to the dataset folder;
    \item Open a terminal and type the following command:\\
        \centerline{\textbf{dotnet run --project SecondAssignment.csproj --configuration Release}}
\end{enumerate}

\subsubsection{Technical characteristics of the computer for measurements}
The execution of the program and the relative measurements were carried out on the following machine:
\begin{itemize}
    \item \textbf{CPU}: Apple M1 Pro;
    \item \textbf{RAM}: 16 GB.
\end{itemize}

\subsection{Introduction to measurements}
The measurements were made through the implementation of three distinct methods, one for each algorithm. \\ \noindent
The measurements were made after retrieving and loading the entire dataset into the graph data structure, as mentioned 
previously. Each method saves the results of the program in a different file, one for each algorithm.\\
\noindent
The single measurement is organized sequentially by running the entire dataset based on the type of algorithm. 
The method \verb|Export2APCSV| applies the \textit{2-Approximation} algorithm to the input graph and proceeds as follows:
\begin{enumerate}
    \item Executes once the 2-Approximation's algorithm to the graph and save the execution time and the approximate solution;
    \item Continue to run the algorithm until the total execution time became greater or equal to 1 second;
    \item Calculate the final execution time as an average between the total sum of the execution time and the number of repetitions of the same algorithm;
    \item Save in append to the output file in \textbf{csv} format.
\end{enumerate}
We also implemented \verb|ExportNearestNeighborCSV| and \verb|ExportClosestInsertionCSV| methods which use \textit{Nearest Neighbor} algorithm and \textit{Closest Insertion}.
\noindent
The output file is saved directly at the end of each algorithm execution keeping the following formatting:
\begin{itemize}
    \item Name of the dataset;
    \item Approximate solution;
    \item Execution time in milliseconds;
    \item Number of executions in one second.
\end{itemize}