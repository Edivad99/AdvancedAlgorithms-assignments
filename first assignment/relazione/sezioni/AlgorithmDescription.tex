\section{Description of the algorithms}

\subsection*{Data structure for a vertex}
The data structure for a vertex was implemented as follows:
\begin{enumerate}
    \item \verb|Name| of the vertex;
    \item \verb|Key| is a kind of weight of a vertex;
    \item \verb|Parent| indicates the parent of a vertex;
    \item \verb|VerticesAdjacent| is a set which contains all adjacent nodes; 
    \item \verb|Visited| that indicates if a node has already been visited or not.
\end{enumerate}
\noindent
The implemented methods were:
\begin{enumerate}
    \item \verb|IsVisited|: returns the value of \verb|Visited|;
    \item \verb|SetVisited|: sets the value of \verb|Visited| to true or false as indicated by the parameter;
    \item \verb|AddAdjacentVertices|: adds an adjacent vertex to the \verb|VerticesAdjacent| of a node;
    \item \verb|RemoveAdjecentVertices|: removes an adjacent vertex to the \verb|VerticesAdjacent|;
    \item \verb|Equals|: checks if two vetrices have the same name.
\end{enumerate}
These operations have been implemented in order to have a constant computational complexity O(1).

\subsection*{Data structure for an edge}
The data structure for am edge was implemented as follows:
\begin{enumerate}
    \item \verb|U| represents one end of the edge;
    \item \verb|V| represents one end of the edge;
    \item \verb|Weight| indicates the weight of the edge.
    \end{enumerate}
\noindent
The implemented methods were:
\begin{enumerate}
    \item \verb|CompareTo|: compares the weight of two different edges and returns an indication of their relative value;
    \item \verb|Equals|: checks if two edges have the same ends and weight.
\end{enumerate}
These operations have been implemented in order to have a constant computational complexity O(1).

\subsection*{Data structure for a graph}
The data structure for a graph was implemented as follows:
\begin{enumerate}
    \item \verb|V| is a set of vertices;
    \item \verb|E| is a collection of edges.
    \end{enumerate}
\noindent
The implemented methods were:
\begin{enumerate}
    \item \verb|AddVertex|: adds a vertex to the graph;
    \item \verb|AddEdge|: adds an edge to the graph;
    \item \verb|RemoveEdge|: removes an edge from the collection \verb|E|;
    \item \verb|GetWeight|: returns the weight of an edge given two vertices;
    \item \verb|LoadFromFileAsync|: creates the graph starting from the selected file.
\end{enumerate}
These operations have been implemented in order to have a constant computational complexity O(1).

\subsection{Kruskal naive}
\subsection{Kruskal with Union-Find}
\subsection{Prim}