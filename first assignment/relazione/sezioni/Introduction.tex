\section{Introduction}

\subsection*{Abstract}
In this brief report we will show comparisons between three algorithms for calculating the \textbf{Minimum Spanning Tree}: Prim's algorithm, Kruskal's algorithm in the naive version and Kruskal's algorithm with Union-Find.

\subsection*{Definition of MST}
Let $V$ be the set of vertices that make up the weighted graph $G$ and let $E$ be the collection of the edges of that graph. With the aim of analyzing the complexity of algorithms, consider $|V| = n$ and $|E| = m$. \\
A minimum spanning tree $T = (V,E')$ is a tree, where the collection of edges $E'$ is a subset of $E$ of a connected, undirected and weighted graph $G = (V,E)$ which interconnects all the vertices $V$ and with the minimum sum of weights. \\
\noindent
One generic MST algorithm is the following: 
\begin{verbatim}
    GENERIC-MST(G)
        A = empty set
        while A does not form a spanning tree
            find an edge (u,v) that is safe for A
            A = A U {(u,v)}
        return A
\end{verbatim}
\noindent
Some definition for MST:
\begin{enumerate}
    \item a cut $(S, V \setminus S)$ of a graph $G = (V, E)$ is a partition of $V$;
    \item an edge $(u,v) \in E$ crosses a cut $(S, V \setminus S)$ if $u\in S$ and $v \in V \setminus S$ or viceversa;
    \item a cut respects a set $A$ of edges if no edge of $A$ crosses the cut;
    \item given a cut, the minumim weight edge that crosses the cut is called \textit{light edge}.
\end{enumerate}
\noindent
To determine if an edge is safe, we will use the following theorem: \\ \noindent
\textbf{Theorem}: Let $G = (V,E)$ be a weighted, connected and undirected graph. Let $A$ be a subset of $E$ of some MST of $G$, let $(S, V \setminus S)$ a cut that respects $A$ and let (u,v) be a light edge for $(S, V \setminus S)$. Then (u,v) is safe for $A$.