\section{Conclusion}

At the end of the work we can say that the results obtained are in line with what this assignment required of us. In fact, the complexity of both \textbf{Stoer-Wagner}'s algorithm and \textbf{Karger and Stein}'s algorithm is almost perfectly overlapped on the expected asymptotic complexity.\\ \\ \noindent
Despite not having implemented many optimizations, thanks to the few improvements made, we have found a clear improvement compared to a first naive implementation of the algorithms making them more efficient. \\  \noindent
In particular, regarding Stoer-Wagner's algorithm, the use of the \verb|SortedSet| data structure for the extraction of the heaviest node in the \verb|StMinCut| function was particularly useful; since this operation has been performed numerous times, it was in fact advantageous to have a function that extracted the maximum in logarithmic time instead of linear. \\ \noindent
As explained also in the slides of the third assignment, we could have implemented the heap as a \textit{Fibonacci Heap}; checking the various input files we have seen that the number of nodes and edges is of the same order of magnitude with a number of edges slightly higher than the number of vertices and therefore the advantage in terms of time would have been difficult to appreciate, risking only to waste time and not complete the two algorithms. \\ \\ \noindent
Talking about the precision of the measurements obtained, we decided to adopt multiple measurements per dataset whose computation took less than one second: in this way we were able to measure the execution time of the algorithms more precisely. \\ \\ \noindent
As was done for the previous assignments, we were able to work well on both algorithms using versioning systems such as GitHub for the C\# code and Overleaf for the report. \\ \\ \noindent
Concluding with regard to the results, it is evident that the best algorithm is Stoer-Wagner based on our code; moreover, the implementation of the algorithms using C\# did not bring us too many problems compared to the other laboratories, given that most of these structures data we have been able to reuse by adapting the code where necessary.