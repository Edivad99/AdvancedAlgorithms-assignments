\section{Description of the algorithms}

\subsection*{Data structure for a vertex}
The data structure for a vertex was implemented as follows:
\begin{enumerate}
    \item \verb|Name| of the vertex;
    \item \verb|Key| is a kind of weight of a vertex;
    \item \verb|Parent| indicates the parent of a vertex;
    \item \verb|VerticesAdjacent| is a set which contains all adjacent nodes; 
    \item \verb|Visited| that indicates if a node has already been visited or not.
\end{enumerate}
\noindent
The implemented methods were:
\begin{enumerate}
    \item \verb|IsVisited|: returns the value of \verb|Visited|;
    \item \verb|SetVisited|: sets the value of \verb|Visited| to true or false as indicated by the parameter;
    \item \verb|AddAdjacentVertices|: adds an adjacent vertex to the \verb|VerticesAdjacent| of a node;
    \item \verb|RemoveAdjecentVertices|: removes an adjacent vertex to the \verb|VerticesAdjacent|;
    \item \verb|Equals|: checks if two vetrices have the same name.
\end{enumerate}
These operations have been implemented in order to have a constant computational complexity O(1).

\subsection*{Data structure for an edge}
The data structure for am edge was implemented as follows:
\begin{enumerate}
    \item \verb|U| represents one end of the edge;
    \item \verb|V| represents one end of the edge;
    \item \verb|Weight| indicates the weight of the edge.
    \end{enumerate}
\noindent
The implemented methods were:
\begin{enumerate}
    \item \verb|CompareTo|: compares the weight of two different edges and returns an indication of their relative value;
    \item \verb|Equals|: checks if two edges have the same ends and weight.
\end{enumerate}
These operations have been implemented in order to have a constant computational complexity O(1).

\subsection*{Data structure for a graph}
The data structure for a graph was implemented as follows:
\begin{enumerate}
    \item \verb|V| is a set of vertices;
    \item \verb|E| is a collection of edges.
    \end{enumerate}
\noindent
The implemented methods were:
\begin{enumerate}
    \item \verb|AddVertex|: adds a vertex to the graph;
    \item \verb|AddEdge|: adds an edge to the graph;
    \item \verb|RemoveEdge|: removes an edge from the collection \verb|E|;
    \item \verb|GetWeight|: returns the weight of an edge given two vertices;
    \item \verb|LoadFromFileAsync|: creates the graph starting from the selected file.
\end{enumerate}
These operations have been implemented in order to have a constant computational complexity O(1).

\subsection{Prim}
\subsubsection{Introduction}
The base version of Prim's Algorithm has a computational complexity of $\mathcal{O}(mn)$.
The base idea is to start with a random vertex and choose, for each interaction, the edge 
that connects with the lowest possible weight the \textit{MST} node tree at the new vertex.\\
The algorithm can be optimized by using the correct data structure.
In fact, by using the Heap we can compute the minimum weight in logarithmic time and the complexity become $\mathcal{O}(mlog(n))$.

\begin{verbatim}
  Prim(G,s)
    for each u in V do
      [u] <- +inf
      parent[u] <- nil
    key[s] <- 0
    Q <- V
    while Q not empty do
      u <- extractMin(Q)
      for each v adjacent to u do
        if v in Q and w(u,v) < key[v] then
          parent[v] <- u
          key[v] <- w(u,v)
\end{verbatim}

\subsubsection{Our implementation}
For the heap structure we used a PriorityQueue that was recently added in .NET 6.0.
The PriorityQueue creates an association between the Vertex object that we pass to it and the weight that vertex has.\\
Our implementation is slightly different from the original one. In fact, instead of filling all the PriorityQueue with 
the vertices, we started by inserting only the starting node with the weight of zero.\\
Then, in the while, we extracting the minimum vertex and we check if the vertex was already visited.
If it was visited we skip it and we extract another one.
Otherwise we mark it as visited, and we starting to check all its adjacent vertices.

\subsection{Kruskal naive}
\subsection{Kruskal with Union-Find}